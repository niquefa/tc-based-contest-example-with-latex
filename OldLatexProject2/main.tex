% Estilo para Colombian Collegiate Programming League - CCPL 2014
% Adaptado de
% Template for a problem -- XXVI Colombian Programming Contest ACIS REDIS 2012

\documentclass[12pt,fleqn]{article}

\usepackage{ccpl2014}

\usepackage[table]{xcolor}
\definecolor{lightgray}{gray}{0.9}

% Choose one of the following styles for input files
\inputstdin    % input through stdin
%\inputfile      % input through named file


\begin{document} 

  % PORTADA
  % Inicializa contador de páginas
  \setcounter{page}{0}

  \begin{center}
    {\huge \vspace{1.5in} Shot Contest 2014\\ 
      \vspace{0.2in}}
    \par
  \end{center}

  \vspace{0.3in}

  \begin{center}
    {\Large 
      1 de noviembre de 2014
    }
    \par
  \end{center}

  \begin{center}
    {\huge \vspace{1in} Poli And UN together to drink}
    \par
  \end{center}

  \begin{center}
    Problem Set.
 
    \par 

    %{\scriptsize (Borrowed from several sources online.)}
  \end{center} 

  \customtoc
  \vfill{}
  
	\newpage
	
	Reglas:
	
\begin{itemize}	
  \item El first solve acarrea un shot para todos los equipos excepto para el que hizo el first solve.
  \item Cada tres envíos malos de un equipo en un mismo problema acarrea un shot para el equipo.
  \item Cada envío aceptado de un problema hace girar la ruleta para sortear shots, si cae el equipo que provocó el giro de la ruleta, se gira nuevamente hasta que caiga otra opción(sólo el primero de cada problema de cada equipo)
  \item Las reglas pueden cambiar según decida el equipo más importante (el de coaches)
\end{itemize}
	
	
	
	
	\begin{center}
    Sitio oficial \url{http://www.alcoholicos-anonimos.org/}
			
		Síguenos en Twitter \href{https://twitter.com/AAGrupoDar}{\url{@AAGrupoDar}}
  \end{center}
	
  \pagebreak

\problem{Sums}{sums}
	We have a String, let’s call it originalWord. Each character of originalWord is either 'a' or 'b'. Timmy claims that he can convert it to finalWord using exactly k moves. In each move, he can either change a single 'a' to a 'b', or change a single 'b' to an 'a'.

You are given the Strings originalWord and finalWord, and the integer value k. Determine whether Timmy may be telling the truth. If there is a possible sequence of exactly k moves that will turn originalWord into finalWord, print "POSSIBLE" (quotes for clarity). Otherwise, print "IMPOSSIBLE".

\subsection* {Input}

The input will consist in several test cases; each test case will consist of a line with originalWord, finalWord and k, each token will be separated by a single space and there will not be leading or trailing spaces. You can assume the following conditions will always be true:
Timmy may change the same letter multiple times. Each time counts as a different move.
originalWord will contain between 1 and 50 characters, inclusive.
finalWord and originalWord will contain the same number of characters.
Each character in originalWord and finalWord will be 'a' or 'b'.
k will be between 1 and 100, inclusive.

\subsection* {Output}

For each test case, print one line with the answer, follow the format on the output example.

\outputnotice

\vspace{12pt}
\begin{minipage}[c]{1\textwidth}%
	\begin{center}
		\begin{tabular}{|l|l|} \hline 
		\begin{minipage}[t]{0.6\textwidth}%
		\bf{Input sample} \\
		\begin{verbatim}
aababba bbbbbbb 2
aabb aabb 1
aaaaabaa bbbbbabb 8
aaa bab 4
aababbabaa abbbbaabab 9

\end{verbatim}
    \end{minipage}%


    \begin{minipage}[t]{0.3\textwidth}%
      \textbf{Output sample} \\      
\begin{verbatim}
IMPOSSIBLE
IMPOSSIBLE
POSSIBLE
POSSIBLE
IMPOSSIBLE


\end{verbatim}
\end{minipage}\\
    \hline
\end{tabular}\end{center}\end{minipage}%

	\problem{Tower}{tower}
	We have a String, let’s call it originalWord. Each character of originalWord is either 'a' or 'b'. Timmy claims that he can convert it to finalWord using exactly k moves. In each move, he can either change a single 'a' to a 'b', or change a single 'b' to an 'a'.

You are given the Strings originalWord and finalWord, and the integer value k. Determine whether Timmy may be telling the truth. If there is a possible sequence of exactly k moves that will turn originalWord into finalWord, print "POSSIBLE" (quotes for clarity). Otherwise, print "IMPOSSIBLE".

\subsection* {Input}

The input will consist in several test cases; each test case will consist of a line with originalWord, finalWord and k, each token will be separated by a single space and there will not be leading or trailing spaces. You can assume the following conditions will always be true:
Timmy may change the same letter multiple times. Each time counts as a different move.
originalWord will contain between 1 and 50 characters, inclusive.
finalWord and originalWord will contain the same number of characters.
Each character in originalWord and finalWord will be 'a' or 'b'.
k will be between 1 and 100, inclusive.

\subsection* {Output}

For each test case, print one line with the answer, follow the format on the output example.

\outputnotice

\vspace{12pt}
\begin{minipage}[c]{1\textwidth}%
	\begin{center}
		\begin{tabular}{|l|l|} \hline 
		\begin{minipage}[t]{0.6\textwidth}%
		\bf{Input sample} \\
		\begin{verbatim}
aababba bbbbbbb 2
aabb aabb 1
aaaaabaa bbbbbabb 8
aaa bab 4
aababbabaa abbbbaabab 9

\end{verbatim}
    \end{minipage}%


    \begin{minipage}[t]{0.3\textwidth}%
      \textbf{Output sample} \\      
\begin{verbatim}
IMPOSSIBLE
IMPOSSIBLE
POSSIBLE
POSSIBLE
IMPOSSIBLE


\end{verbatim}
\end{minipage}\\
    \hline
\end{tabular}\end{center}\end{minipage}%

	\problem{Songs}{songs}
	We have a String, let’s call it originalWord. Each character of originalWord is either 'a' or 'b'. Timmy claims that he can convert it to finalWord using exactly k moves. In each move, he can either change a single 'a' to a 'b', or change a single 'b' to an 'a'.

You are given the Strings originalWord and finalWord, and the integer value k. Determine whether Timmy may be telling the truth. If there is a possible sequence of exactly k moves that will turn originalWord into finalWord, print "POSSIBLE" (quotes for clarity). Otherwise, print "IMPOSSIBLE".

\subsection* {Input}

The input will consist in several test cases; each test case will consist of a line with originalWord, finalWord and k, each token will be separated by a single space and there will not be leading or trailing spaces. You can assume the following conditions will always be true:
Timmy may change the same letter multiple times. Each time counts as a different move.
originalWord will contain between 1 and 50 characters, inclusive.
finalWord and originalWord will contain the same number of characters.
Each character in originalWord and finalWord will be 'a' or 'b'.
k will be between 1 and 100, inclusive.

\subsection* {Output}

For each test case, print one line with the answer, follow the format on the output example.

\outputnotice

\vspace{12pt}
\begin{minipage}[c]{1\textwidth}%
	\begin{center}
		\begin{tabular}{|l|l|} \hline 
		\begin{minipage}[t]{0.6\textwidth}%
		\bf{Input sample} \\
		\begin{verbatim}
aababba bbbbbbb 2
aabb aabb 1
aaaaabaa bbbbbabb 8
aaa bab 4
aababbabaa abbbbaabab 9

\end{verbatim}
    \end{minipage}%


    \begin{minipage}[t]{0.3\textwidth}%
      \textbf{Output sample} \\      
\begin{verbatim}
IMPOSSIBLE
IMPOSSIBLE
POSSIBLE
POSSIBLE
IMPOSSIBLE


\end{verbatim}
\end{minipage}\\
    \hline
\end{tabular}\end{center}\end{minipage}%

	\problem{Cow}{cow}
	We have a String, let’s call it originalWord. Each character of originalWord is either 'a' or 'b'. Timmy claims that he can convert it to finalWord using exactly k moves. In each move, he can either change a single 'a' to a 'b', or change a single 'b' to an 'a'.

You are given the Strings originalWord and finalWord, and the integer value k. Determine whether Timmy may be telling the truth. If there is a possible sequence of exactly k moves that will turn originalWord into finalWord, print "POSSIBLE" (quotes for clarity). Otherwise, print "IMPOSSIBLE".

\subsection* {Input}

The input will consist in several test cases; each test case will consist of a line with originalWord, finalWord and k, each token will be separated by a single space and there will not be leading or trailing spaces. You can assume the following conditions will always be true:
Timmy may change the same letter multiple times. Each time counts as a different move.
originalWord will contain between 1 and 50 characters, inclusive.
finalWord and originalWord will contain the same number of characters.
Each character in originalWord and finalWord will be 'a' or 'b'.
k will be between 1 and 100, inclusive.

\subsection* {Output}

For each test case, print one line with the answer, follow the format on the output example.

\outputnotice

\vspace{12pt}
\begin{minipage}[c]{1\textwidth}%
	\begin{center}
		\begin{tabular}{|l|l|} \hline 
		\begin{minipage}[t]{0.6\textwidth}%
		\bf{Input sample} \\
		\begin{verbatim}
aababba bbbbbbb 2
aabb aabb 1
aaaaabaa bbbbbabb 8
aaa bab 4
aababbabaa abbbbaabab 9

\end{verbatim}
    \end{minipage}%


    \begin{minipage}[t]{0.3\textwidth}%
      \textbf{Output sample} \\      
\begin{verbatim}
IMPOSSIBLE
IMPOSSIBLE
POSSIBLE
POSSIBLE
IMPOSSIBLE


\end{verbatim}
\end{minipage}\\
    \hline
\end{tabular}\end{center}\end{minipage}%

	\problem{Changer}{changer}
	We have a String, let’s call it originalWord. Each character of originalWord is either 'a' or 'b'. Timmy claims that he can convert it to finalWord using exactly k moves. In each move, he can either change a single 'a' to a 'b', or change a single 'b' to an 'a'.

You are given the Strings originalWord and finalWord, and the integer value k. Determine whether Timmy may be telling the truth. If there is a possible sequence of exactly k moves that will turn originalWord into finalWord, print "POSSIBLE" (quotes for clarity). Otherwise, print "IMPOSSIBLE".

\subsection* {Input}

The input will consist in several test cases; each test case will consist of a line with originalWord, finalWord and k, each token will be separated by a single space and there will not be leading or trailing spaces. You can assume the following conditions will always be true:
Timmy may change the same letter multiple times. Each time counts as a different move.
originalWord will contain between 1 and 50 characters, inclusive.
finalWord and originalWord will contain the same number of characters.
Each character in originalWord and finalWord will be 'a' or 'b'.
k will be between 1 and 100, inclusive.

\subsection* {Output}

For each test case, print one line with the answer, follow the format on the output example.

\outputnotice

\vspace{12pt}
\begin{minipage}[c]{1\textwidth}%
	\begin{center}
		\begin{tabular}{|l|l|} \hline 
		\begin{minipage}[t]{0.6\textwidth}%
		\bf{Input sample} \\
		\begin{verbatim}
aababba bbbbbbb 2
aabb aabb 1
aaaaabaa bbbbbabb 8
aaa bab 4
aababbabaa abbbbaabab 9

\end{verbatim}
    \end{minipage}%


    \begin{minipage}[t]{0.3\textwidth}%
      \textbf{Output sample} \\      
\begin{verbatim}
IMPOSSIBLE
IMPOSSIBLE
POSSIBLE
POSSIBLE
IMPOSSIBLE


\end{verbatim}
\end{minipage}\\
    \hline
\end{tabular}\end{center}\end{minipage}%

	\problem{Vocabulary}{vocabulary}
	We have a String, let’s call it originalWord. Each character of originalWord is either 'a' or 'b'. Timmy claims that he can convert it to finalWord using exactly k moves. In each move, he can either change a single 'a' to a 'b', or change a single 'b' to an 'a'.

You are given the Strings originalWord and finalWord, and the integer value k. Determine whether Timmy may be telling the truth. If there is a possible sequence of exactly k moves that will turn originalWord into finalWord, print "POSSIBLE" (quotes for clarity). Otherwise, print "IMPOSSIBLE".

\subsection* {Input}

The input will consist in several test cases; each test case will consist of a line with originalWord, finalWord and k, each token will be separated by a single space and there will not be leading or trailing spaces. You can assume the following conditions will always be true:
Timmy may change the same letter multiple times. Each time counts as a different move.
originalWord will contain between 1 and 50 characters, inclusive.
finalWord and originalWord will contain the same number of characters.
Each character in originalWord and finalWord will be 'a' or 'b'.
k will be between 1 and 100, inclusive.

\subsection* {Output}

For each test case, print one line with the answer, follow the format on the output example.

\outputnotice

\vspace{12pt}
\begin{minipage}[c]{1\textwidth}%
	\begin{center}
		\begin{tabular}{|l|l|} \hline 
		\begin{minipage}[t]{0.6\textwidth}%
		\bf{Input sample} \\
		\begin{verbatim}
aababba bbbbbbb 2
aabb aabb 1
aaaaabaa bbbbbabb 8
aaa bab 4
aababbabaa abbbbaabab 9

\end{verbatim}
    \end{minipage}%


    \begin{minipage}[t]{0.3\textwidth}%
      \textbf{Output sample} \\      
\begin{verbatim}
IMPOSSIBLE
IMPOSSIBLE
POSSIBLE
POSSIBLE
IMPOSSIBLE


\end{verbatim}
\end{minipage}\\
    \hline
\end{tabular}\end{center}\end{minipage}%

	\problem{Test}{test}
	We have a String, let’s call it originalWord. Each character of originalWord is either 'a' or 'b'. Timmy claims that he can convert it to finalWord using exactly k moves. In each move, he can either change a single 'a' to a 'b', or change a single 'b' to an 'a'.

You are given the Strings originalWord and finalWord, and the integer value k. Determine whether Timmy may be telling the truth. If there is a possible sequence of exactly k moves that will turn originalWord into finalWord, print "POSSIBLE" (quotes for clarity). Otherwise, print "IMPOSSIBLE".

\subsection* {Input}

The input will consist in several test cases; each test case will consist of a line with originalWord, finalWord and k, each token will be separated by a single space and there will not be leading or trailing spaces. You can assume the following conditions will always be true:
Timmy may change the same letter multiple times. Each time counts as a different move.
originalWord will contain between 1 and 50 characters, inclusive.
finalWord and originalWord will contain the same number of characters.
Each character in originalWord and finalWord will be 'a' or 'b'.
k will be between 1 and 100, inclusive.

\subsection* {Output}

For each test case, print one line with the answer, follow the format on the output example.

\outputnotice

\vspace{12pt}
\begin{minipage}[c]{1\textwidth}%
	\begin{center}
		\begin{tabular}{|l|l|} \hline 
		\begin{minipage}[t]{0.6\textwidth}%
		\bf{Input sample} \\
		\begin{verbatim}
aababba bbbbbbb 2
aabb aabb 1
aaaaabaa bbbbbabb 8
aaa bab 4
aababbabaa abbbbaabab 9

\end{verbatim}
    \end{minipage}%


    \begin{minipage}[t]{0.3\textwidth}%
      \textbf{Output sample} \\      
\begin{verbatim}
IMPOSSIBLE
IMPOSSIBLE
POSSIBLE
POSSIBLE
IMPOSSIBLE


\end{verbatim}
\end{minipage}\\
    \hline
\end{tabular}\end{center}\end{minipage}%

	\problem{Dancing}{dancing}
	We have a String, let’s call it originalWord. Each character of originalWord is either 'a' or 'b'. Timmy claims that he can convert it to finalWord using exactly k moves. In each move, he can either change a single 'a' to a 'b', or change a single 'b' to an 'a'.

You are given the Strings originalWord and finalWord, and the integer value k. Determine whether Timmy may be telling the truth. If there is a possible sequence of exactly k moves that will turn originalWord into finalWord, print "POSSIBLE" (quotes for clarity). Otherwise, print "IMPOSSIBLE".

\subsection* {Input}

The input will consist in several test cases; each test case will consist of a line with originalWord, finalWord and k, each token will be separated by a single space and there will not be leading or trailing spaces. You can assume the following conditions will always be true:
Timmy may change the same letter multiple times. Each time counts as a different move.
originalWord will contain between 1 and 50 characters, inclusive.
finalWord and originalWord will contain the same number of characters.
Each character in originalWord and finalWord will be 'a' or 'b'.
k will be between 1 and 100, inclusive.

\subsection* {Output}

For each test case, print one line with the answer, follow the format on the output example.

\outputnotice

\vspace{12pt}
\begin{minipage}[c]{1\textwidth}%
	\begin{center}
		\begin{tabular}{|l|l|} \hline 
		\begin{minipage}[t]{0.6\textwidth}%
		\bf{Input sample} \\
		\begin{verbatim}
aababba bbbbbbb 2
aabb aabb 1
aaaaabaa bbbbbabb 8
aaa bab 4
aababbabaa abbbbaabab 9

\end{verbatim}
    \end{minipage}%


    \begin{minipage}[t]{0.3\textwidth}%
      \textbf{Output sample} \\      
\begin{verbatim}
IMPOSSIBLE
IMPOSSIBLE
POSSIBLE
POSSIBLE
IMPOSSIBLE


\end{verbatim}
\end{minipage}\\
    \hline
\end{tabular}\end{center}\end{minipage}%

	\problem{Machine}{machine}
	We have a String, let’s call it originalWord. Each character of originalWord is either 'a' or 'b'. Timmy claims that he can convert it to finalWord using exactly k moves. In each move, he can either change a single 'a' to a 'b', or change a single 'b' to an 'a'.

You are given the Strings originalWord and finalWord, and the integer value k. Determine whether Timmy may be telling the truth. If there is a possible sequence of exactly k moves that will turn originalWord into finalWord, print "POSSIBLE" (quotes for clarity). Otherwise, print "IMPOSSIBLE".

\subsection* {Input}

The input will consist in several test cases; each test case will consist of a line with originalWord, finalWord and k, each token will be separated by a single space and there will not be leading or trailing spaces. You can assume the following conditions will always be true:
Timmy may change the same letter multiple times. Each time counts as a different move.
originalWord will contain between 1 and 50 characters, inclusive.
finalWord and originalWord will contain the same number of characters.
Each character in originalWord and finalWord will be 'a' or 'b'.
k will be between 1 and 100, inclusive.

\subsection* {Output}

For each test case, print one line with the answer, follow the format on the output example.

\outputnotice

\vspace{12pt}
\begin{minipage}[c]{1\textwidth}%
	\begin{center}
		\begin{tabular}{|l|l|} \hline 
		\begin{minipage}[t]{0.6\textwidth}%
		\bf{Input sample} \\
		\begin{verbatim}
aababba bbbbbbb 2
aabb aabb 1
aaaaabaa bbbbbabb 8
aaa bab 4
aababbabaa abbbbaabab 9

\end{verbatim}
    \end{minipage}%


    \begin{minipage}[t]{0.3\textwidth}%
      \textbf{Output sample} \\      
\begin{verbatim}
IMPOSSIBLE
IMPOSSIBLE
POSSIBLE
POSSIBLE
IMPOSSIBLE


\end{verbatim}
\end{minipage}\\
    \hline
\end{tabular}\end{center}\end{minipage}%

	\problem{Power}{power}
	We have a String, let’s call it originalWord. Each character of originalWord is either 'a' or 'b'. Timmy claims that he can convert it to finalWord using exactly k moves. In each move, he can either change a single 'a' to a 'b', or change a single 'b' to an 'a'.

You are given the Strings originalWord and finalWord, and the integer value k. Determine whether Timmy may be telling the truth. If there is a possible sequence of exactly k moves that will turn originalWord into finalWord, print "POSSIBLE" (quotes for clarity). Otherwise, print "IMPOSSIBLE".

\subsection* {Input}

The input will consist in several test cases; each test case will consist of a line with originalWord, finalWord and k, each token will be separated by a single space and there will not be leading or trailing spaces. You can assume the following conditions will always be true:
Timmy may change the same letter multiple times. Each time counts as a different move.
originalWord will contain between 1 and 50 characters, inclusive.
finalWord and originalWord will contain the same number of characters.
Each character in originalWord and finalWord will be 'a' or 'b'.
k will be between 1 and 100, inclusive.

\subsection* {Output}

For each test case, print one line with the answer, follow the format on the output example.

\outputnotice

\vspace{12pt}
\begin{minipage}[c]{1\textwidth}%
	\begin{center}
		\begin{tabular}{|l|l|} \hline 
		\begin{minipage}[t]{0.6\textwidth}%
		\bf{Input sample} \\
		\begin{verbatim}
aababba bbbbbbb 2
aabb aabb 1
aaaaabaa bbbbbabb 8
aaa bab 4
aababbabaa abbbbaabab 9

\end{verbatim}
    \end{minipage}%


    \begin{minipage}[t]{0.3\textwidth}%
      \textbf{Output sample} \\      
\begin{verbatim}
IMPOSSIBLE
IMPOSSIBLE
POSSIBLE
POSSIBLE
IMPOSSIBLE


\end{verbatim}
\end{minipage}\\
    \hline
\end{tabular}\end{center}\end{minipage}%

	\problem{Chess}{chess}
	We have a String, let’s call it originalWord. Each character of originalWord is either 'a' or 'b'. Timmy claims that he can convert it to finalWord using exactly k moves. In each move, he can either change a single 'a' to a 'b', or change a single 'b' to an 'a'.

You are given the Strings originalWord and finalWord, and the integer value k. Determine whether Timmy may be telling the truth. If there is a possible sequence of exactly k moves that will turn originalWord into finalWord, print "POSSIBLE" (quotes for clarity). Otherwise, print "IMPOSSIBLE".

\subsection* {Input}

The input will consist in several test cases; each test case will consist of a line with originalWord, finalWord and k, each token will be separated by a single space and there will not be leading or trailing spaces. You can assume the following conditions will always be true:
Timmy may change the same letter multiple times. Each time counts as a different move.
originalWord will contain between 1 and 50 characters, inclusive.
finalWord and originalWord will contain the same number of characters.
Each character in originalWord and finalWord will be 'a' or 'b'.
k will be between 1 and 100, inclusive.

\subsection* {Output}

For each test case, print one line with the answer, follow the format on the output example.

\outputnotice

\vspace{12pt}
\begin{minipage}[c]{1\textwidth}%
	\begin{center}
		\begin{tabular}{|l|l|} \hline 
		\begin{minipage}[t]{0.6\textwidth}%
		\bf{Input sample} \\
		\begin{verbatim}
aababba bbbbbbb 2
aabb aabb 1
aaaaabaa bbbbbabb 8
aaa bab 4
aababbabaa abbbbaabab 9

\end{verbatim}
    \end{minipage}%


    \begin{minipage}[t]{0.3\textwidth}%
      \textbf{Output sample} \\      
\begin{verbatim}
IMPOSSIBLE
IMPOSSIBLE
POSSIBLE
POSSIBLE
IMPOSSIBLE


\end{verbatim}
\end{minipage}\\
    \hline
\end{tabular}\end{center}\end{minipage}%

	\problem{Haar}{haar}
	We have a String, let’s call it originalWord. Each character of originalWord is either 'a' or 'b'. Timmy claims that he can convert it to finalWord using exactly k moves. In each move, he can either change a single 'a' to a 'b', or change a single 'b' to an 'a'.

You are given the Strings originalWord and finalWord, and the integer value k. Determine whether Timmy may be telling the truth. If there is a possible sequence of exactly k moves that will turn originalWord into finalWord, print "POSSIBLE" (quotes for clarity). Otherwise, print "IMPOSSIBLE".

\subsection* {Input}

The input will consist in several test cases; each test case will consist of a line with originalWord, finalWord and k, each token will be separated by a single space and there will not be leading or trailing spaces. You can assume the following conditions will always be true:
Timmy may change the same letter multiple times. Each time counts as a different move.
originalWord will contain between 1 and 50 characters, inclusive.
finalWord and originalWord will contain the same number of characters.
Each character in originalWord and finalWord will be 'a' or 'b'.
k will be between 1 and 100, inclusive.

\subsection* {Output}

For each test case, print one line with the answer, follow the format on the output example.

\outputnotice

\vspace{12pt}
\begin{minipage}[c]{1\textwidth}%
	\begin{center}
		\begin{tabular}{|l|l|} \hline 
		\begin{minipage}[t]{0.6\textwidth}%
		\bf{Input sample} \\
		\begin{verbatim}
aababba bbbbbbb 2
aabb aabb 1
aaaaabaa bbbbbabb 8
aaa bab 4
aababbabaa abbbbaabab 9

\end{verbatim}
    \end{minipage}%


    \begin{minipage}[t]{0.3\textwidth}%
      \textbf{Output sample} \\      
\begin{verbatim}
IMPOSSIBLE
IMPOSSIBLE
POSSIBLE
POSSIBLE
IMPOSSIBLE


\end{verbatim}
\end{minipage}\\
    \hline
\end{tabular}\end{center}\end{minipage}%

	\problem{Sequence}{sequence}
	We have a String, let’s call it originalWord. Each character of originalWord is either 'a' or 'b'. Timmy claims that he can convert it to finalWord using exactly k moves. In each move, he can either change a single 'a' to a 'b', or change a single 'b' to an 'a'.

You are given the Strings originalWord and finalWord, and the integer value k. Determine whether Timmy may be telling the truth. If there is a possible sequence of exactly k moves that will turn originalWord into finalWord, print "POSSIBLE" (quotes for clarity). Otherwise, print "IMPOSSIBLE".

\subsection* {Input}

The input will consist in several test cases; each test case will consist of a line with originalWord, finalWord and k, each token will be separated by a single space and there will not be leading or trailing spaces. You can assume the following conditions will always be true:
Timmy may change the same letter multiple times. Each time counts as a different move.
originalWord will contain between 1 and 50 characters, inclusive.
finalWord and originalWord will contain the same number of characters.
Each character in originalWord and finalWord will be 'a' or 'b'.
k will be between 1 and 100, inclusive.

\subsection* {Output}

For each test case, print one line with the answer, follow the format on the output example.

\outputnotice

\vspace{12pt}
\begin{minipage}[c]{1\textwidth}%
	\begin{center}
		\begin{tabular}{|l|l|} \hline 
		\begin{minipage}[t]{0.6\textwidth}%
		\bf{Input sample} \\
		\begin{verbatim}
aababba bbbbbbb 2
aabb aabb 1
aaaaabaa bbbbbabb 8
aaa bab 4
aababbabaa abbbbaabab 9

\end{verbatim}
    \end{minipage}%


    \begin{minipage}[t]{0.3\textwidth}%
      \textbf{Output sample} \\      
\begin{verbatim}
IMPOSSIBLE
IMPOSSIBLE
POSSIBLE
POSSIBLE
IMPOSSIBLE


\end{verbatim}
\end{minipage}\\
    \hline
\end{tabular}\end{center}\end{minipage}%

	\problem{Jaguars}{jaguars}
	We have a String, let’s call it originalWord. Each character of originalWord is either 'a' or 'b'. Timmy claims that he can convert it to finalWord using exactly k moves. In each move, he can either change a single 'a' to a 'b', or change a single 'b' to an 'a'.

You are given the Strings originalWord and finalWord, and the integer value k. Determine whether Timmy may be telling the truth. If there is a possible sequence of exactly k moves that will turn originalWord into finalWord, print "POSSIBLE" (quotes for clarity). Otherwise, print "IMPOSSIBLE".

\subsection* {Input}

The input will consist in several test cases; each test case will consist of a line with originalWord, finalWord and k, each token will be separated by a single space and there will not be leading or trailing spaces. You can assume the following conditions will always be true:
Timmy may change the same letter multiple times. Each time counts as a different move.
originalWord will contain between 1 and 50 characters, inclusive.
finalWord and originalWord will contain the same number of characters.
Each character in originalWord and finalWord will be 'a' or 'b'.
k will be between 1 and 100, inclusive.

\subsection* {Output}

For each test case, print one line with the answer, follow the format on the output example.

\outputnotice

\vspace{12pt}
\begin{minipage}[c]{1\textwidth}%
	\begin{center}
		\begin{tabular}{|l|l|} \hline 
		\begin{minipage}[t]{0.6\textwidth}%
		\bf{Input sample} \\
		\begin{verbatim}
aababba bbbbbbb 2
aabb aabb 1
aaaaabaa bbbbbabb 8
aaa bab 4
aababbabaa abbbbaabab 9

\end{verbatim}
    \end{minipage}%


    \begin{minipage}[t]{0.3\textwidth}%
      \textbf{Output sample} \\      
\begin{verbatim}
IMPOSSIBLE
IMPOSSIBLE
POSSIBLE
POSSIBLE
IMPOSSIBLE


\end{verbatim}
\end{minipage}\\
    \hline
\end{tabular}\end{center}\end{minipage}%

	\problem{Kingdom}{kingdom}
	We have a String, let’s call it originalWord. Each character of originalWord is either 'a' or 'b'. Timmy claims that he can convert it to finalWord using exactly k moves. In each move, he can either change a single 'a' to a 'b', or change a single 'b' to an 'a'.

You are given the Strings originalWord and finalWord, and the integer value k. Determine whether Timmy may be telling the truth. If there is a possible sequence of exactly k moves that will turn originalWord into finalWord, print "POSSIBLE" (quotes for clarity). Otherwise, print "IMPOSSIBLE".

\subsection* {Input}

The input will consist in several test cases; each test case will consist of a line with originalWord, finalWord and k, each token will be separated by a single space and there will not be leading or trailing spaces. You can assume the following conditions will always be true:
Timmy may change the same letter multiple times. Each time counts as a different move.
originalWord will contain between 1 and 50 characters, inclusive.
finalWord and originalWord will contain the same number of characters.
Each character in originalWord and finalWord will be 'a' or 'b'.
k will be between 1 and 100, inclusive.

\subsection* {Output}

For each test case, print one line with the answer, follow the format on the output example.

\outputnotice

\vspace{12pt}
\begin{minipage}[c]{1\textwidth}%
	\begin{center}
		\begin{tabular}{|l|l|} \hline 
		\begin{minipage}[t]{0.6\textwidth}%
		\bf{Input sample} \\
		\begin{verbatim}
aababba bbbbbbb 2
aabb aabb 1
aaaaabaa bbbbbabb 8
aaa bab 4
aababbabaa abbbbaabab 9

\end{verbatim}
    \end{minipage}%


    \begin{minipage}[t]{0.3\textwidth}%
      \textbf{Output sample} \\      
\begin{verbatim}
IMPOSSIBLE
IMPOSSIBLE
POSSIBLE
POSSIBLE
IMPOSSIBLE


\end{verbatim}
\end{minipage}\\
    \hline
\end{tabular}\end{center}\end{minipage}%

 	\problem{Sea}{sea}
	We have a String, let’s call it originalWord. Each character of originalWord is either 'a' or 'b'. Timmy claims that he can convert it to finalWord using exactly k moves. In each move, he can either change a single 'a' to a 'b', or change a single 'b' to an 'a'.

You are given the Strings originalWord and finalWord, and the integer value k. Determine whether Timmy may be telling the truth. If there is a possible sequence of exactly k moves that will turn originalWord into finalWord, print "POSSIBLE" (quotes for clarity). Otherwise, print "IMPOSSIBLE".

\subsection* {Input}

The input will consist in several test cases; each test case will consist of a line with originalWord, finalWord and k, each token will be separated by a single space and there will not be leading or trailing spaces. You can assume the following conditions will always be true:
Timmy may change the same letter multiple times. Each time counts as a different move.
originalWord will contain between 1 and 50 characters, inclusive.
finalWord and originalWord will contain the same number of characters.
Each character in originalWord and finalWord will be 'a' or 'b'.
k will be between 1 and 100, inclusive.

\subsection* {Output}

For each test case, print one line with the answer, follow the format on the output example.

\outputnotice

\vspace{12pt}
\begin{minipage}[c]{1\textwidth}%
	\begin{center}
		\begin{tabular}{|l|l|} \hline 
		\begin{minipage}[t]{0.6\textwidth}%
		\bf{Input sample} \\
		\begin{verbatim}
aababba bbbbbbb 2
aabb aabb 1
aaaaabaa bbbbbabb 8
aaa bab 4
aababbabaa abbbbaabab 9

\end{verbatim}
    \end{minipage}%


    \begin{minipage}[t]{0.3\textwidth}%
      \textbf{Output sample} \\      
\begin{verbatim}
IMPOSSIBLE
IMPOSSIBLE
POSSIBLE
POSSIBLE
IMPOSSIBLE


\end{verbatim}
\end{minipage}\\
    \hline
\end{tabular}\end{center}\end{minipage}%


  
 \end{document}
