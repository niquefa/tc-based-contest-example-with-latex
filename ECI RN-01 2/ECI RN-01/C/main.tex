Vocaloids $Gumi$, $Ia$, and $Mayu$ love singing. They decided to make an album composed of S songs. Each of the S songs must be sung by at least one of the three Vocaloids. It is allowed for some songs to be sung by any two, or even all three Vocaloids at the same time. The number of songs sang by Gumi, Ia, and Mayu must be $gumi$, $ia$, and $mayu$, respectively. 

They soon realized that there are many ways of making the album. Two albums are considered different if there is a song that is sung by a different set of Vocaloids. Let X be the number of possible albums. Since the number X can be quite large, compute and print the number (X modulo 1,000,000,007). $S$ will be between 1 and 50, inclusive. $gumi$, $ia$ and $mayu$ will be each between 1 and $S$, inclusive.

For the first sample test case:  In this case, there are 3 songs on the album. And Gumi, Ia, Mayu will each sing one song. There are 3*2*1 = 6 ways how to choose which Vocaloid sings which song. For the second sample test case: Gumi will sing all three songs. Ia and Mayu can each choose which one song they want to sing. For the third sample test case:
It is not possible to record 50 songs if each Vocaloid can only sing 10 of them.


\subsection* {Input}

The input will contain several test case, each in a line. The integer values $S$, $gumi$, $ia$, and $mayu$.

\subsection* {Output}

Print a line per test case, the solution to the problem.

\outputnotice

\vspace{12pt}
\begin{minipage}[c]{1\textwidth}%
	\begin{center}
		\begin{tabular}{|l|l|} \hline 
		\begin{minipage}[t]{0.6\textwidth}%
		\bf{Input sample} \\
		\begin{verbatim}
3 1 1 1
3 3 1 1
50 10 10 10
18 12 8 9
50 25 25 25
3 3 1 1
2 1 1 1

\end{verbatim}
    \end{minipage}%


    \begin{minipage}[t]{0.3\textwidth}%
      \textbf{Output sample} \\      
\begin{verbatim}
6
9
0
81451692
198591037
9
6

\end{verbatim}
\end{minipage}\\
    \hline
\end{tabular}\end{center}\end{minipage}%
