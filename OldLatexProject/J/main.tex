A number n taken to the falling factorial power k is defined as n*(n-1)*...*(n-k+1). We will denote it by n+++k. For example, 7+++3=7*6*5=210. By definition, n+++1=n.

We will now continue this definition to the non-positive values of k using the following fact: (n-k)*(n+++k)=n+++(k+1), or, in other words, n+++k=(n+++(k+1))/(n-k). It is directly derived from the above definition.

By using it, we find:

n+++0=n+++1/(n-0)=1,\\
n+++(-1)=n+++0/(n+1)=1/(n+1),\\
n+++(-2)=1/(n+1)/(n+2),\\
and, in general, n+++(-k)=1/(n+1)/(n+2)/.../(n+k).\\
For example, 3+++(-1)=1/4=0.25, 2+++(-3)=1/3/4/5=1/60=0.016666...\\
Given a positive integer n $(1<=n<=10)$ and an integer k $(-5<=n<=5)$, find and print a real number containing the value of n taken to the falling factorial power of k.



\subsection* {Input}

The input will consist in several test cases; each test case will consist of a line with n and k.

\subsection* {Output}

For each test case, print one line with the answer, follow the format on the output example. The answer must be rounded and printed with sex decimal positions.

\outputnotice

\vspace{12pt}
\begin{minipage}[c]{1\textwidth}%
	\begin{center}
		\begin{tabular}{|l|l|} \hline 
		\begin{minipage}[t]{0.6\textwidth}%
		\bf{Input sample} \\
		\begin{verbatim}
7 3
10 1
5 0
3 -1
2 -3

\end{verbatim}
    \end{minipage}%


    \begin{minipage}[t]{0.3\textwidth}%
      \textbf{Output sample} \\      
\begin{verbatim}
210.000000
10.000000
1.000000
0.250000
0.016667


\end{verbatim}
\end{minipage}\\
    \hline
\end{tabular}\end{center}\end{minipage}%
