A sentence is called dancing if its first letter is uppercase and the case of each subsequent letter is the opposite of the previous letter. Spaces should be ignored when determining the case of a letter. For example, "A b Cd" is a dancing sentence because the first letter ('A') is uppercase, the next letter ('b') is lowercase, the next letter ('C') is uppercase, and the next letter ('d') is lowercase.

You will be given a sentence. Turn the sentence into a dancing sentence by changing the cases of the letters where necessary. All spaces in the original sentence must be preserved.

\subsection* {Input}

The input will consist in several test cases; each test case will consist of a line with the given sentence. You can assume the following conditions will always be true:
The given sentence will contain between 1 and 50 characters, inclusive.
Each character in the given sentence will be a letter ('A'-'Z','a'-'z') or a space (' ').
The given sentence will contain at least one letter ('A'-'Z','a'-'z').

\subsection* {Output}

For each test case, print one line with the answer, follow the format on the output example.

\outputnotice

\vspace{12pt}
\begin{minipage}[c]{1\textwidth}%
	\begin{center}
		\begin{tabular}{|l|l|} \hline 
		\begin{minipage}[t]{0.45\textwidth}%
		\bf{Input sample} \\
		\begin{verbatim}
This is a dancing sentence
 This is a dancing sentence 
aaaaaaaaaaa
z

\end{verbatim}
    \end{minipage}%


    \begin{minipage}[t]{0.45\textwidth}%
      \textbf{Output sample} \\      
\begin{verbatim}
ThIs Is A dAnCiNg SeNtEnCe
 ThIs Is A dAnCiNg SeNtEnCe 
AaAaAaAaAaA
Z

\end{verbatim}
\end{minipage}\\
    \hline
\end{tabular}\end{center}\end{minipage}%
